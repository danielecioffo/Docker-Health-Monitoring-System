\section{REST Interface}
This REST APIs were generated using \textit{Swagger Editor} and the functionalities implemented are:

\begin{itemize}
	\item \textbf{GET}:
		\subitem \textit{containers/} : Retrieve the list of all containers with their informations \footnote{The Swagger\_server computes the total number of actives hosts, but if some of them go down the GET operation retrieves only the ones that respond within a lap of time equal to 4 seconds. If all of the hosts answer to the request in a shorter time, the REST interface stops waiting for answers and send the list of all containers back to the user}. The \textit{response} is a list of containers containing for each one its hostname, its name, its status and if it is monitored or not. 
	\item \textbf{POST}:
		\subitem \textit{containers/\{hostname\}/\{containerName\}} : Monitor the container specified in the path
	\item \textbf{DELETE}:
		\subitem \textit{containers/\{hostname\}/\{containerName\}} : Unmonitor the container specified in the path
	\item \textbf{PUT}:
		\subitem \textit{/threshold} : Update the packet loss threshold used in the Agent. The value is specified in the body of the request and its type is \textit{double}.
\end{itemize}

\noindent The REST Interface makes use also of a model for specifying the structure of a Container, which is the following\footnote{the asterisk indicates that the attribute must be set}:

\begin{figure}[H]
	\begin{subfigure}{\textwidth}
	\centering
		\includegraphics[width=0.9\linewidth]{img/container.png} 
	\end{subfigure}
\end{figure}