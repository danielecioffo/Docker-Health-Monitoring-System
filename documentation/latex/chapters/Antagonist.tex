\section{Antagonist}
In order to test the health monitoring system a small program (the antagonist) has been developed to stop containers or cause some artificial packet loss.
More precisely, the antagonist was implemented in two different modules. 

\noindent One of the two modules is implemented in each host on a docker container, starting from a customized Ubuntu image. In this container there is the python interpreter which is used to execute the antagonist.py file.
This first module periodically attacks one of the local docker containers, simulating a stop of that container (leveraging the Docker API for python).

\noindent The second module has been implemented inside each dummy containers, which we will discuss in detail in the next chapter. This module allows you to simulate network problems on a certain container, through the use of the Netem library. The simulated packet loss percentage is periodically changed, through the use of this command: \emph{tc qdisc add dev eth0 root netem loss XX\%}, with XX the packet loss value.